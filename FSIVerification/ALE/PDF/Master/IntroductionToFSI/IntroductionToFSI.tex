
The Fluid-Structure Interaction problem can be seen in many parts of nature, from large industrial engineering complexes to the smallest blood vessels in the human body. A terrifying example is the collapse of the Tacoma Narrows Bridge that collapsed in 1940 only two months after being opened. The collapse was du to aero-elastic fluttering from strong winds. At the smaller scale, inter-cranial aneurysms which are balloon shaped geometries often occurring where a blood vessel splits into to two parts, due to weak vessel wall. Bursting of one of these aneurysms in the skull can have fatale consequences. We can therefore easily see the need to model these problems. The apparent difficulty however in both these problems is when fluid velocities reach speeds which gives turbulence. The solvers need then to handle these turbulences. The main goal of this master thesis is to build a framework to solve the FSI problem, looking at different approaches and schemes. The framework will be validated and verified through well known benchmarks.  


