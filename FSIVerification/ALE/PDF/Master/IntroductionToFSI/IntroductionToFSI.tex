
The Fluid-Structure Interaction problem can be observed all around us in nature, from large industrial engineering complexes to the smallest blood vessels in the human body. A large scale example is the collapse of the Tacoma Narrows Bridge that collapsed in 1940 only two months after being opened. The collapse was due to aero-elastic fluttering from strong winds. No human life was lost in the collapse, but a cocker spaniel name Tubby left in a car was not so lucky. The construction of windmills are a second example of the Fluid-Structure Interaction problem. Todays windmills are rigid and hence giving a big difference in density between fluid and structure, $ \rho_s >> \rho_f $. The structure will therefore only give rise to small deformations. However applying FSI to hemodynamics( dynamics of blood flow ) deems more challenging. One FSI hemodynamic problem are inter-cranial aneurysms, which are balloon shaped geometries often occurring where a blood vessel splits into two parts, due to weak vessel walls. Bursting of one of these aneurysms in the skull can have fatale consequences. With fluid and structure densities more equal than the previous example, the structure has an elastic character giving under the right circumstances large deformations. The blood flow also transitions to turbulent flow. This combination gives the need for a rigid stabile solver. Therefore the main goal of this master thesis is to build a framework to solve the FSI problem, investigating different approaches and schemes. The framework will be validated and verified using MMS, companying a wide range of benchmarks.  


